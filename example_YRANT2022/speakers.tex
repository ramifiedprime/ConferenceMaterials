\newspeaker{RachelNewton}{Rachel}{Newton}{KCL}{racheldominicanewton@gmail.com}{Distribution of genus groups of abelian number fields}{Let K be a number field and let L/K be an abelian extension. The genus field of L/K is the largest extension of L which is unramified at all places of L and abelian as an extension of K. The genus group is its Galois group over L, which is a quotient of the class group of L, and the genus number is the size of the genus group. We study the distribution of genus numbers as one varies over abelian extensions L/K with fixed Galois group. This is joint work with Christopher Frei and Daniel Loughran.}
\newspeaker{JamesNewton}{James}{Newton}{Univeristy of Oxford}{newton@maths.ox.ac.uk}{Modularity of elliptic curves over CM fields}{I will discuss some recent progress towards establishing modularity of elliptic curves over CM number fields (particularly imaginary quadratic fields). The new results I will talk about are joint work with Ana Caraiani.}
\newspeaker{JessicaFintzen}{Jessica}{Fintzen}{Universit\"at Bonn, University of Cambridge, Duke University}{jessica@math.uni-bonn.de}{Representations of p-adic groups}{The Langlands program is a far-reaching collection of
conjectures that relate different areas of mathematics including number
theory and representation theory. A fundamental problem on the
representation theory side of the Langlands program is the construction
of all (irreducible, smooth, complex) representations of $p$-adic groups.
I will provide an overview of our understanding of the representation
theory of $p$-adic groups with a focus on recent developments. I will also
briefly discuss applications to other areas.}

\newspeaker{RobinAmmon}{Robin}{Ammon}{University of Glasgow}{r.ammon.1@research.gla.ac.uk}{Cohen--Lenstra Heuristics: Distribution of Class Groups of Random Number Fields }{The ideal class group of an algebraic number field captures a great deal of information about the arithmetic of the number field, making it a central object in algebraic number theory. Despite its great importance, however, not much is known about its structure in general. In order to shed some more light on the behaviour of class groups, H. Cohen and H. W. Lenstra Jr. studied their statistics in families of number fields, and building on their observations made influential conjectures about the distribution of class groups. These have later been extended and refined, but except for very few special cases remain unsolved to date.     In this brief introduction to the Cohen--Lenstra heuristics, I will explain the ideas of arithmetic statistics behind the probabilistic approach to class groups, and will discuss the general principle for the distribution of random algebraic objects that underlies Cohen and Lenstra's at first curious seeming conjectures and which may eventually lead to a better understanding of class groups and related objects.}
\newspeaker{DavidKurniadiAngdinata}{David Kurniadi}{Angdinata}{LSGNT}{ucahdka@ucl.ac.uk}{Formalisation of elliptic curves in Lean}{Can a computer check the proof of Fermat’s last theorem? Certainly, it needs to know the basics of modular curves to state the modularity theorem, and even then it has to arduously learn much of 20th century arithmetic geometry to get anywhere in Wiles’ proof. In this talk, I will introduce the Lean theorem prover – a potential solution to this in the far future – and the components in its mathematical library relevant to algebraic number theory. I will also describe the ongoing attempt to formalise the arithmetic of elliptic curves to begin one side of the story.}
\newspeaker{JamieBell}{Jamie}{Bell}{UCL}{ucahell@ucl.ac.uk}{$p^{\infty}$ – Selmer ranks and Complex Multiplication}{The $p^{\infty}$ – Selmer rank of an elliptic curve is a number related to the rank which is often easier to deal with than the rank itself. Using his work on the parity conjecture, Česnavičius proved in 2012 that for curves with complex multiplication (those with lots of endomorphisms), this is even. I will present an alternative proof, and discuss how it can be generalised to certain abelian varieties.}
\newspeaker{DominikBullach}{Dominik}{Bullach}{King's College London}{dominik.bullach@kcl.ac.uk}{Cyclotomic units and the scarcity of Euler systems}{Ever since their introduction, Euler systems have played an important role in spectacular advances in arithmetic geometry. In this talk, I will discuss Coleman's Conjecture (proved in recent joint work with Burns, Daoud, and Seo), which gives a precise description of the set of all classical Euler systems over the rationals. If time permits, I will also indicate striking consequences of these ideas towards the conjecture of Bloch and Kato over general number fields.  }
\newspeaker{SvenCats}{Sven}{Cats}{University of Cambridge }{cats.sven@gmail.com}{Higher Descents on Elliptic Curves}{Computing the \textit{$n$-Selmer group} of an elliptic curve $E$ over the rational numbers $\mathbb{Q}$ gives an upper bound for the rank of $E(\mathbb{Q})$. The $n$-Selmer group parameterises isomorphism classes of \textit{$n$-coverings} of $E$ that are everywhere locally soluble; by an \textit{explicit $n$-descent} we mean computing equations for such $n$-coverings. For certain applications, such as finding generators of large height for $E(\mathbb{Q})$, it is useful to be able to take $n$ as large as possible. In this talk we discuss a way to achieve such higher descents: to combine an $n$- and an $(n+1)$-descent into an $n \cdot (n+1)$-descent.}
\newspeaker{DiegoChicharro}{Diego}{Chicharro}{UCL/LSGNT}{diego.chicharro.21@ucl.ac.uk}{Multiple Zeta Values}{The values of the Riemann zeta function at even positive integers have been known to be rational multiples of powers of pi for a long time, but not so much is known for odd integers, although these special values seem to contain deep arithmetic information. Since generalisations usually lead to new insights, one can try to study a ``multidimensional’’ version of the zeta function, thus getting a large family of special numbers called multiple zeta values. It turns out that these values satisfy many Q-linear relations, and conjecturally we know them all, but the proof of this remains out of reach. In this talk, we will discuss how these numbers are related to the geometry of the projective line minus three points, and explain how the "motivic" point of view allows us to prove partial results in that direction. Namely, we can bound below the number of linearly independent relations between MZVs by the number we expect.}
\newspeaker{LewisCombes}{Lewis}{Combes}{University of Sheffield}{lmcombes1@sheffield.ac.uk}{Computing with mod $p$ Selmer groups}{Selmer groups attached to Galois representations are (conjecturally) steeped in valuable information. In this talk we explain how to compute certain Selmer groups attached to mod p representations, and some interpretations of associated arithmetic information. }
\newspeaker{JoshDrewitt}{Josh}{Drewitt}{The University of Nottingham}{joshua.drewitt1@nottingham.ac.uk}{ Applications of real-analytic modular forms.}{The space of real-analytic modular forms was recently introduced by Francis Brown. One reason this space is of great interest is because it contains or intersects various classes of important modular objects, such as classical modular forms, weakly anti-holomorphic forms and Maass wave forms. Therefore, the space of real-analytic modular forms can be viewed as a unifying tool for all these individual subspaces. The purpose of this talk is to introduce you to this space and to demonstrate a few of its interesting properties and applications. We give examples of real-analytic modular forms, discuss the period polynomials and L-functions related to this space, and see how these forms connect to the study of string theory in physics.  }
\newspeaker{FabioFerri}{Fabio}{Ferri}{University of Exeter}{ff263@exeter.ac.uk}{On reduction steps to Leopoldt's conjecture}{ Let p be a rational prime and let L/K be a Galois extension of number fields with Galois group G. Under some hypotheses, we show that Leopoldt’s conjecture at p for certain proper intermediate fields of L/K implies Leopoldt’s conjecture at p for L; a crucial tool will be the theory of norm relations in Q[G]. We also consider relations between the Leopoldt defects at p of intermediate extensions of L/K. Joint with Henri Johnston.}
\newspeaker{BenceForras}{Bence}{Forrás}{Universität Duisburg-Essen}{bence.forras@uni-due.de}{Integrality of smoothed $p$-adic Artin $L$-functions}{We consider a one-dimensional admissible $p$-adic Lie extension of number fields, and introduce a smoothed version of the equivariant $S$-truncated $p$-adic Artin $L$-function. By integrality we mean that the $L$-function lives inside some maximal order in the total ring of quotients of the Iwasawa algebra associated with the extension. Building on the works of Nichifor–Palvannan, Johnston–Nickel, and Lau, and assuming the Equivariant Iwasawa Main Conjecture, integrality of this $L$-function has been verified in general. In the special case when the extension is $p$-abelian, a stronger version of integrality is known unconditionally.}
\newspeaker{SamFrengley}{Sam}{Frengley}{University of Cambridge}{stf32@cam.ac.uk}{Pairs of elliptic curves with isomorphic mod 12 Galois representations}{We will discuss the construction of infinite families of pairs of elliptic curves defined over $\mathbb{Q}$ with isomorphic mod $12$ Galois representations. Our approach is to compute explicit birational models for the surfaces which parametrise such pairs of elliptic curves. This extends previous work of Chen and Fisher where it is assumed that the underlying isomorphisms of $12$-torsion subgroups respect the Weil pairing.  If time allows we may discuss applications to splittings of Jacobians of genus $2$ curves.}
\newspeaker{StevanGajovic}{Stevan}{Gajović}{Max Planck Institute Bonn}{gajovic@mpim-bonn.mpg.de}{Reverse-engineered Diophantine equations}{Solving some Diophantine equations can be a very difficult problem. In this talk, we will consider an opposite problem - can we construct Diophantine equations of a certain type with a prescribed set of solutions? More precisely, we consider equations of a shape $f(x)=y^n$. For a given finite set $S$ of integral powers, we look for a polynomial $f$ with integral coefficients such that only integral powers that appear when $f$ is evaluated at integers are precisely those in $S$. }
\newspeaker{JohannesGirsch}{Johannes}{Girsch}{Imperial College}{johannes.girsch@live.de}{The Doubling Method in Algebraic Families}{Local constants are an important concept in the complex representation theory of reductive $p$-adic groups, for example they are pivotal in the formulation of the Local Langlands correspondence. In recent years there has been progress in defining such constants for modular representations or in even more general settings. For example, Moss was able to define $\gamma$-factors for representations of $\textup{GL}_n(\mathbb{Q}_p)$ with coefficients in general noetherian rings and subsequently together with Helm was able to prove a converse theorem, which was crucial for the proof of the Local Langlands correspondence in families for $\textup{GL}_n$. The aim of this talk is to show how one can extend the Doubling Method of Piateski-Shapiro and Rallis to families of representations of classical groups. In this setting we will introduce and prove a rationality result for the Doubling Zeta integrals. Subsequently we will show that these zeta integrals satisfy a functional equation from which one obtains $\gamma$-factors.}
\newspeaker{JakobGlas}{Jakob}{Glas}{IST Austria}{jglas@ist.ac.at}{Diagonal Cubic Forms over $\mathbb{F}_q(t)$: Part II}{Understanding the set of rational points of a variety over a global field is one of the cornerstones of number theory. We report on our recent work on the distribution of rational points on diagonal cubic surfaces and fourfolds over $\mathbb{F}_q(t)$ via the circle method. Finally, we shall indicate how this leads to progress on the cubic Waring's problem over $\mathbb{F}_q(t)$. }
\newspeaker{StevenGroen}{Steven}{Groen}{University of Warwick}{stevengroen95@gmail.com}{p-torsion in characteristic p of Jacobians of Artin-Schreier curves }{In characteristic 0, the geometric torsion of an Abelian varieties is well understood; there is only one possibility that occurs. However, in characteristic p, multiplication by p is inseparable, giving many possibilities for A[p]. We are particularly interested in the case when A is the Jacobian of an Artin-Schreier curve Y. Obtaining information about the p-torsion of Jac(Y) involves an analysis of the so-called Cartier operator on the space of regular differentials of Y. We survey some known results and prove that the ranks of powers of the Cartier operator lie within certain bounds, which coincide in some special cases.}
\newspeaker{LeonhardHochfilzer}{Leonhard}{Hochfilzer}{University of Göttingen}{leonhard.hochfilzer@gmail.com}{Diagonal Cubic Forms over $\mathbb{F}_q(t)$: Part I}{Understanding the set of rational points of a variety over a global field is one of the cornerstones of number theory. We report on our recent work on the distribution of rational points on diagonal cubic surfaces and fourfolds over $\mathbb{F}_q(t)$ via the circle method. Finally, we shall indicate how this leads to progress on the cubic Waring's problem over $\mathbb{F}_q(t)$. }
\newspeaker{LukasKofler}{Lukas}{Kofler}{University of Cambridge}{lukas.kofler@maths.cam.ac.uk}{A plectic Langlands-Rapoport conjecture}{The Langlands-Rapoport conjecture predicts the number of points modulo p of a Shimura variety. We will look at modular curves before explaining its general formalism using motives, gerbes, and affine Deligne-Lusztig varieties. Now known in many cases due to Kisin, the conjecture is a key step in the Langlands program. The plectic conjectures of Nekovar and Scholl predict that certain Shimura varieties are equipped with additional structure of motivic origin. We will sketch how this might lead to a refinement of the Langlands-Rapoport conjecture.}
\newspeaker{AlexandrosKonstantinou}{Alexandros}{Konstantinou}{PhD Student - UCL}{zcahkon@ucl.ac.uk}{A quest for deriving local formulae}{The calculation of the Mordell-Weil rank of an Abelian variety $A/K$ remains an open problem. Unable to find a precise value for it, and in view of predictions made by the parity conjecture, one is motivated to express its parity in terms of invariants which depend on Abelian varieties defined over the various completions of $K$. The aim of this talk is twofold: (1) We first introduce a new method for expressing the rank parity of a certain class of abelian varieties using local data, and (2) granted the finiteness of the Tate - Shafarevich group, we show how one can obtain a local expression for the rank parity of the Jacobian variety of a curve. The content of this talk is based on joint work with V. Dokchitser, H. Green and A. Morgan.}
\newspeaker{LorenzoLaPorta}{Lorenzo}{La Porta}{King's College London}{lorenzo.1.la\_porta@kcl.ac.uk}{Generalised $\theta$ operators on some unitary Shimura varieties}{The theory of the classical theta operator was instrumental in Edixhoven's proof of the weight part of Serre's modularity conjecture. Because of this, much work has been devoted to extending the construction of this operator to other Shimura varieties, with an eye towards generalisations of Serre's conjecture, or to gain insight in the Langlands programme $(\textnormal{mod}\,p)$ in a broader sense. My goal is to present the construction of a new ``generalised'' theta operator that seems to produce exactly the weight shifts that one would expect from a representation-theoretic viewpoint and ties in neatly with the theory of generalised Hasse invariants of Boxer and Goldring-Koskivirta. I will focus on the case of Picard modular surfaces as a key example, review the construction of the ``ordinary'' theta operator (inspired by the work Eischen, Mantovan et al. and Goren-de Shalit) and show how one can obtain a similar differential operator on the closure of the 1-dimensional Ekedahl-Oort stratum.}
\newspeaker{ZeyuLiu}{Zeyu}{Liu}{University of California, San Diego}{zeliu@ucsd.edu}{From de Rham prismatic crystals to nearly de Rham representations.}{We explain how to read off the information of a de Rham prismatic crystal over $\mathbb{O}_K$  from its associated nearly de Rham representation.}
\newspeaker{SebastianMonnet}{Sebastian}{Monnet}{UCL}{sebastian.monnet.21@ucl.ac.uk}{Parametrising number fields}{Imagine you're an undergraduate learning about fields for the first time. It might not be immediately obvious how to classify quadratic extensions of the rational numbers. An elementary argument shows that quadratic fields are in natural one-to-one correspondence with squarefree integers; we say that squarefree integers \emph{parametrise} quadratic number fields. This parametrisation is very useful. Among other things, it lets us answer questions like "How many quadratic number fields are there with discriminant in $[-X, X]$?", for positive real numbers $X$. This idea has been generalised to cubic number fields by Delone-Faddeev, and to quartics and quintics by Bhargava. In this talk, I will sketch the Delone-Faddeev parametrisation of cubic number fields, as well as its application by Davenport-Heilbronn to counting cubic fields.}
\newspeaker{AshvniNarayanan}{Ashvni}{Narayanan}{Imperial College London}{a.narayanan20@imperial.ac.uk}{Formalisation of $p$-adic $L$-functions in Lean}{L-functions are a well-studied part of number theory. I am working on the formalization of p-adic L-functions (attached to a Dirichlet character) in terms of a Bernoulli measure in the mathematical library of a computer language, Lean. I will also try to explain the generalization of this definition to p-adic L-functions for modular forms. }
\newspeaker{MartinOrtizRamirez}{Martin}{Ortiz Ramirez}{LSGNT}{martin.ortiz.21@ucl.ac.uk}{Prismatic F-crystals}{Prismatic F-crystals are an object in p-adic Hodge theory recently introduced by Bhatt and Scholze which nicely packages some of the previous objects in (integral) p-adic Hodge theory. I will explain what this means, making explicit the link with $(\phi,\Gamma)$-modules.}
\newspeaker{OttoOverkamp}{Otto}{Overkamp}{University of Oxford}{otto.overkamp@maths.ox.ac.uk}{Jacobians of geometrically reduced curves and their N\'eron models}{We study the structure of Jacobians of geometrically reduced curves over arbitrary (i. e., not necessarily perfect) fields. We apply our results to prove two conjectures due to Bosch-L\"utkebohmert-Raynaud about the existence of N\'eron models and N\'eron lft-models over excellent Dedekind schemes in the special case of Jacobians of geometrically reduced curves.}
\newspeaker{JessePajwani}{Jesse}{Pajwani}{Imperial College London}{j.pajwani19@ic.ac.uk}{The Massey Vanishing Conjecture}{The Massey Vanishing conjecture (of Minac--Tan) is a recent (2013) conjecture concerning "Massey products" of Galois cohomology classes. The conjecture has connections to many different areas, including knot theory, the Milnor conjecture, and the inverse Galois problem. As well as this, there are many ways of approaching the problem, including changing the problem to a rational points problem for certain "splitting varieties".  In this talk, I'll introduce what Massey products are and introduce the problem as stated, before highlighting its connection to other problems. I'll then end by giving a quick survey of the results in the area, ending with the recent proof of Harpaz--Wittenberg of the "weak Massey vanishing conjecture in total degree 1 for number fields".}
\newspeaker{SunilKumarPasupulati}{Sunil Kumar}{Pasupulati}{Indian Institute of Science Education and Research Thiruvananthapuram}{sunil4960016@iisertvm.ac.in}{On the Existence of Euclidean ideal class in quartic, quadratic, and cubic  extensions}{In 1979, Lenstra  introduced the definition of the  Euclidean ideal which is a generalization of Euclidean domain.\begin{definition} Let $R$ be a Dedekind domain and $\mathbb{I}$ be the set of non zero integral ideals of $R$.  If $C$ is an ideal of $R,$ then it is called Euclidean if there exists a function $\Psi:\mathbb{I} \to \mathbb{N}$, such that for every $I \in \mathbb{I} $ and  $x\in I^{-1}C \setminus C$  there exist a  $y\in C$ such that $$ \Psi\left( (x-y)IC^{-1} \right) < \Psi (I).$$ \end{definition} Lenstra established that for a number field $K$ with $\textnormal{rank}(\mathcal{O}_K^{\times})\geq 1$, the number ring $\mathcal{O}_K$ contains a Euclidean ideal if and only if the class group $Cl_K$  is cyclic, provided GRH holds. Several authors worked towards removing the assumption of GRH. In this talk, I  prove the existence of the Euclidean ideal class in abelian quartic fields. As a corollary,  I will prove that a certain class biquadratic field with class number two has a Euclidean ideal class. I also discuss the existence of a Euclidean ideal class in certain cubic and quadratic extensions. This is joint work with Srilakshmi Krishnamoorthy.}
\newspeaker{AndrewPearce-Crump}{Andrew}{Pearce-Crump}{University of York}{aepc501@york.ac.uk}{The Generalised Shanks' Conjecture}{In this talk we give a brief outline of the history of moments of the Riemann zeta function, $\zeta (s)$, and a related problem, known as Shanks' Conjecture. This states that $\zeta ' (\rho)$ is real and positive in the mean, where the sum is over the non-trivial zeros of $\zeta (s)$. We discuss a generalisation of this result, which states that $\zeta ^{(n)} (\rho)$ is real and positive/negative in the mean, depending on whether n is odd/even, and outline a proof of the result.}
\newspeaker{JoshPimm}{Josh}{Pimm}{University of Nottingham}{Joshua.Pimm@nottingham.ac.uk}{Shimura Correspondences and Their Generalisations}{In 1973 Shimura gave the first systematic correspondence of modular forms from half-integer weight to integer weight, a theta lift taking weight k/2 (k in 2Z+1) forms to forms of weight k-1. In 1975 Shintani established a map in the opposite direction. In this talk, I will discuss the maps of Shimura and Shintani, as well as the generalisations of Borcherds, Zagier, Li-Zemel and Duke-Jenkins. In particular, I will talk about maps between weakly holomorphic and vector-valued forms, and the appearance of the twisted trace function in fourier coefficients of lifted forms. We will also see the general compatibility with Hecke operators exhibited by these maps.}
\newspeaker{JennyRoberts}{Jenny}{Roberts}{University of Bristol}{jenny.roberts@bristol.ac.uk}{Newform Eisenstein congruences of local origin}{The theory of Eisenstein congruences dates back to Ramanujan's surprising discovery that the Fourier coefficients of the discriminant function are congruent to the 11th power divisor sum modulo 691.  This observation can be explained via the congruence of two modular forms of weight 12 and level 1; the discriminant function and the Eisenstein series, E12. We explore a generalisation of this result to newforms of weight k>2, prime level and non-trivial character.  }
\newspeaker{KaterinaSanticola}{Katerina}{Santicola}{University of Warwick}{Katerina.Santicola@warwick.ac.uk}{Reverse Engineered Diophantine Equations over $\mathbb{Q}$}{Let $S=\{b_1,...,b_k\}\subseteq \mathcal{P}_{\mathbb{Z}}$ be a finite set of perfect powers. Is there a polynomial $f_S\in \mathbb{Z}[x]$ such that $f_S(\mathbb{Z})\cap \mathcal{P}_{\mathbb{Z}}=S$? This question was recently answered in the affirmative by Stevan Gajovic. We will show that the construction can be extended to $\mathbb{Q}$. Namely, we show for $S=\{b_1,...,b_k\}\subseteq \mathcal{P}_{\mathbb{Q}}$, there exists $f_S\in \mathbb{Z}[x]$ such that $f_S(\mathbb{Q})\cap \mathcal{P}_{\mathbb{Q}}=S$.}
\newspeaker{SamStreeter}{Sam}{Streeter}{University of Bristol}{sam.streeter@bristol.ac.uk}{Semi-integral points}{In this talk we will explore two bridges from the familiar world of rational points to the mysterious realm of integral points. If the bridges are sufficiently sturdy, one might hope to wheel over rational points machinery into integral land and discover something exciting. Even if that’s too much to ask, these bridges, called Campana points and Darmon points, offer us a geometric way to study solutions to equations in powerful numbers and perfect powers, the focus of major results and unsolved problems in number theory.}
\newspeaker{JaySwar}{Jay}{Swar}{University of Oxford}{swar@maths.ox.ac.uk}{Generalizing p-adic L-functions on moduli spaces of Galois representations}{One central highway between the theory of p-adic L-functions and Diophantine-geometric consequences is built upon their role in controlling/annihilating Selmer groups. In this talk, we'll introduce the viewpoint of p-adic L-functions in non-commutative Iwasawa theory, and see that similar phenomena exist in the theory of derived Selmer schemes.}
\newspeaker{GuillemTarrach}{Guillem}{Tarrach}{University of Cambridge}{gg457@cam.ac.uk}{$S$-arithmetic cohomology and $p$-adic automorphic forms  }{In the last few decades, the theory of $p$-adic modular forms has seen many applications to different problems in number theory. This theory is well-understood, its central objects of study being overconvergent $p$-adic modular forms. However, when attempting to generalize the theory to more general reductive groups, the picture is less clear. For example, there are several different proposed definitions for spaces of $p$-adic automorphic forms, such as overconvergent and completed cohomology. In this talk I will introduce a different proposal, based on the study of the cohomology of $p$-arithmetic groups with coefficients in $p$-adic locally analytic representations. }
\newspeaker{MohamedTawfik}{Mohamed}{Tawfik}{King's College London}{mohamed.tawfik@kcl.ac.uk}{Brauer-Manin obstructions on Kummer surfaces of products of CM elliptic curves}{We construct a family of pairs of CM elliptic curves such that the Kummer surface of the product of each pair has non-trivial transcendental Brauer group. Then we construct the transcendental Brauer group of the Kummer surface using the work of Skorobogatov and Zarhin by finding a generator of the group. Finally, we prove the existence of a transcendental Brauer-Manin obstruction to weak approximation on the Kummer surfaces.}
\newspeaker{JustinTrias}{Justin}{Trias}{Imperial College }{jtrias@ic.ac.uk }{Towards an integral theta correspondence for type II dual pairs}{The classical local theta correspondence for p-adic reductive dual pairs defines a bijection between prescribed subsets of irreducible smooth complex representations coming from two groups (H,H'), forming a dual pair in a symplectic group. Alberto Mínguez extended this result for type II dual pairs to representations with coefficients in an algebraically closed field of characteristic l as long as the characteristic l does not divide the pro-orders of H and H'. For coefficients rings like Z[1/p], we explain how to build a theory in families for type II dual pairs that is compatible with reduction to residue fields of the base coefficient ring, where central to this approach is the integral Bernstein centre. We translate some weaker properties of the classical correspondence, such as compatibility with supercuspidal support, as a morphism between the integral Bernstein centres of H and H' and interpret it for the Weil representation. In general, we only know that this morphism is finite though we may expect it to be surjective. This would result in a closed immersion between the associated affine schemes as well as a correspondence between characters of the Bernstein centre. This is current work with Gil Moss.}
\newspeaker{HarkaranUppal}{Harkaran}{Uppal}{University of Bath}{Hsu20@bath.ac.uk}{Hasse principle and cubic hypersurfaces.}{In this talk, I will introduce the Brauer-Manin obstruction. I will explain to how calculate the obstruction, to understand the failure of the Hasse principle for rational and integral points on certain cubic hypersurfaces.}
\newspeaker{ArtWaeterschoot}{Art}{Waeterschoot}{KU Leuven }{Art.waeterschoot@kuleuven.be}{wild models of curves via nonarchimedean geometry}{Given a smooth variety over the p-adics, one can construct a p-adic analytification as a space of geometric valuations following Berkovich, so that GAGA principles hold. The result is a nice locally compact Hausdorff topological space that is an inductive limit of certain simplicial complexes, called skeleta, that are combinatorially obtained by reductions of the variety modulo p.  Given any generically étale morphism of smooth varieties over a complete discretely valued field, one can compare certain canonical skeleta in the analytifications in terms of piecewise linear functions related to logarithmic differentials. Applying this to base change morphisms of curves, one can recover results on models of curves and wild ramification.}
\newspeaker{DmitriWhitmore}{Dmitri}{Whitmore}{University of Cambridge}{dw517@cam.ac.uk}{The Taylor--Wiles method for reductive groups }{The Langlands philosophy predicts a map from automorphic representations to representations of the Galois group $\Gamma$ of a number field. In this talk we discuss automorphy lifting, which is a way to show a representation of $\Gamma$ arises from an automorphic representation. Such a theorem is proved via the Taylor--Wiles method; we explore a generalisation allowing representations with small (residual) image. We conclude with an application to modularity of some elliptic curves, building upon the work of Boxer—Calegari--Gee--Pilloni.}
\newspeaker{HannekeWiersema}{Hanneke}{Wiersema}{University of Cambridge}{hw600@cam.ac.uk}{Modularity in the partial weight one case }{The strong form of Serre's conjecture states that a two-dimensional mod p representation of the absolute Galois group of Q arises from a modular form of a specific weight, level and character. Serre considered modular forms of weight at least 2, but in 1992 Edixhoven refined this conjecture to include weight one modular forms. In this talk we discuss analogues of Edixhoven's refinement for Galois representations of totally real fields, extending recent work of Diamond–Sasaki.}
\newspeaker{CameronWilson}{Cameron}{Wilson}{University of Glasgow}{c.wilson.6@research.gla.ac.uk}{General Bilinear Forms over the Jacobi Symbol (and their application to rational point problems)}{tba}
\newspeaker{SiqiYang}{Siqi}{Yang}{LSGNT}{siqi.yang.21@ucl.ac.uk}{Computing weight one modular forms}{ Weight one classical modular forms are special in the sense that they don't have a (precise) dimension formula and mod p weight one modular forms don't always lift to characteristic zero. These bring difficulties in computing the weight one modular forms, especially in characteristic p cases. This short talk will introduce the method that is given by Buzzard and further discussed by Schaeffer. We consider modular forms of higher weight to give an upper bound for the dimension of the space of weight one modular forms. Also, the associated Galois representations help us give a lower bound if we consider representations with certain properties.}
\newspeaker{YuanYang}{Yuan}{Yang}{LSGNT}{yy.lsgnt@gmail.com}{Is LMFDB wrong? A small story between me and LMFDB.}{The whole story began when I was trying to find an explicit example for Gross-Zagier formula, for the modular curve $X_0(11)$ and imaginary quadratic field $\mathbb{Q}(\sqrt{-2})$. The Heegner point of discriminant $-8$, $P$, whose two j-invariants both equal $8000$, which are both rational integer and this makes $P$ a rational point on $X_0(11)$. According to LMFDB, $X_0(11)$ is an elliptic curve \texttt{11.a2} who has Mordell-Weil rank $0$ and as a result $P$ is a torsion point. But this contradicts Gross-Zagier formula, as the Gross-Zagier formula predicts $P$ has non-zero height! Who is wrong in the whole process?  }
\newspeaker{SadiahZahoor}{Sadiah}{Zahoor}{University of Sheffield }{Sadiahzahoor@gmail.com }{ Congruences related to modular forms}{The objective of this talk to introduce and prove congruences related to modular forms.  Our main motivation to study such congruences is the fact that these can be used to give evidence in support of well known conjectures in Number Theory like Birch Swinnerton-Dyer  conjecture and the Tamagawa conjecture of Bloch and Kato. In order to do so, we use the theory half-integer weight modular forms and the arithmetic significance of their Fourier coefficients as captured by the Waldpurger’s theorem.  We will show how mod p congruences between modular forms in general transcend to similar mod p congruences between half-integer weight modular forms with slightly shifted Fourier coefficients. We try to use a recipe in our proof that can be easily generalised to Hilbert modular forms over totally real quadratic fields of narrow class number 1.}
